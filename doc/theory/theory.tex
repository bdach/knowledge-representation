\documentclass[11pt,a4paper]{article}
\usepackage{fullpage}
\usepackage[T1]{fontenc} 
\usepackage[utf8]{inputenc}
\usepackage{amsmath}
\usepackage{amssymb}
\usepackage{amsthm}
\usepackage{float}
\usepackage{tabularx}
\usepackage[hidelinks]{hyperref}
\usepackage[polish]{babel}

\setlength{\parindent}{0cm}
\setlength{\parskip}{2mm}

\newtheorem{defn}{Definicja}

\begin{document}

\title{Reprezentacja wiedzy \\
\Large{
    Projekt nr 4. --- Programy działań z akcjami współbieżnymi \\
    Podstawy teoretyczne
}}
\author{
    Bartłomiej Dach \and
    Jacek Dziwulski \and
    Tymon Felski \and
    Jędrzej Fijałkowski \and
    Filip Grajek \and
    Maciej Grzeszczak \and
    Michał Kołodziej \and
    Piotr Piwowarski \and
    Mateusz Rymuszka \and
    Piotr Wolski
}
\maketitle

\section{Założenia}

Dana jest klasa systemów dynamicznych spełniających następujące warunki:

\begin{enumerate}
    \item Prawo inercji.
    \item Niedeterminizm.
    \item W języku kwerend występują akcje złożone (zbiory co najwyżej $k$ akcji atomowych), w~języku akcji jedynie akcje atomowe.
    \item Pełna informacja o~wszystkich akcjach atomowych i~wszystkich ich skutkach bezpośrednich.
    \item Z~każdą akcją atomową związany jest jej warunek wstępny (ew. \textsc{True}) i~końcowy (efekt akcji).
    \item Wykonywane są jedynie akcje bezkonfliktowe (żadne dwie akcje składowe nie~mogą mieć wspólnych zmiennych, na~które w~żadnym stanie mają wpływ).
    \item Wynikiem akcji złożonej jest suma skutków wszystkich składowych akcji bezkonfliktowych.
    \item Akcje mogą być niewykonalne w~pewnych stanach; jeśli akcja jest niewykonywalna, to~każda akcja ją zawierająca też jest niewykonalna.
    \item Dopuszczalny jest opis częściowy zarówno stanu początkowego, jak i~pewnych stanów wynikających z~wykonań sekwencji akcji.
\end{enumerate}

\begin{defn}
    \textbf{Programem $P$ działań} jest ciąg $P = (A_1, \dots,A_n)$ akcji złożonych.
\end{defn}

\begin{defn}
    Program $P = (A_1, \dots, A_n)$ jest \textbf{realizowalny}, jeśli wszystkie akcje złożone $A_i$ są~wykonalne.
\end{defn}

Celem projektu jest opracowanie i~zaimplementowanie języka akcji dla~specyfikacji podanej klasy systemów dynamicznych oraz~odpowiadającego mu języka kwerend zapewniającego uzyskanie odpowiedzi na następujące zapytania:

\begin{enumerate}
    \item Czy podany program $P$ działań jest możliwy do~realizacji zawsze/kiedykolwiek ze stanu początkowego?
    \item Czy wykonanie programu $P$ w~stanie początkowym działań prowadzi zawsze/kiedykolwiek do~osiągnięcia celu~$\gamma$?
    \item Czy cel~$\gamma$ jest osiągalny ze~stanu początkowego?
\end{enumerate}

% TODO: definicja struktury, formuły

\section{Język akcji}

\subsection{Składnia języka}

\subsection{Semantyka języka}

\section{Język kwerend}

\subsection{Składnia języka}

\subsection{Semantyka języka}

\section{Przykładowe kwerendy}

\end{document}