\documentclass[11pt,a4paper]{article}
\usepackage{fullpage}
\usepackage[T1]{fontenc} 
\usepackage[utf8]{inputenc}
\usepackage{amsmath}
\usepackage{amssymb}
\usepackage{amsthm}
\usepackage{float}
\usepackage{tabularx}
\usepackage[hidelinks]{hyperref}
\usepackage[polish]{babel}
\usepackage{tikz}
\usetikzlibrary{positioning}

\setlength{\parindent}{0cm}
\setlength{\parskip}{2mm}

\newtheorem{defn}{Definicja}
\newtheorem{theorem}{Twierdzenie}

% definicje operatorów w zdaniach

\DeclareMathOperator{\Initially}{\textbf{\textit{initially}}}
\DeclareMathOperator{\After}{\textbf{\textit{after}}}
\DeclareMathOperator{\Observable}{\textbf{\textit{observable}}}
\DeclareMathOperator{\Causes}{\textbf{\textit{causes}}}
\DeclareMathOperator{\If}{\textbf{\textit{if}}}
\DeclareMathOperator{\Impossible}{\textbf{\textit{impossible}}}
\DeclareMathOperator{\Releases}{\textbf{\textit{releases}}}
\DeclareMathOperator{\Always}{\textbf{\textit{always}}}
\DeclareMathOperator{\Sometimes}{\textbf{\textit{sometimes}}}
\DeclareMathOperator{\Noninertial}{\textbf{\textit{noninertial}}}

\DeclareMathOperator{\Executable}{\textbf{\textit{executable}}}
\DeclareMathOperator{\Accessible}{\textbf{\textit{accessible}}}
\DeclareMathOperator{\Possibly}{\textbf{\textit{possibly}}}
\DeclareMathOperator{\Necessary}{\textbf{\textit{necessary}}}

\def\consequence{\mathrel|\joinrel\approx}

\begin{document}

\title{Reprezentacja wiedzy \\
\Large{
    Projekt nr 4. --- Programy działań z akcjami współbieżnymi \\
    Podstawy teoretyczne
}}
\author{
    Bartłomiej Dach \and
    Jacek Dziwulski \and
    Tymon Felski \and
    Jędrzej Fijałkowski \and
    Filip Grajek \and
    Maciej Grzeszczak \and
    Michał Kołodziej \and
    Piotr Piwowarski \and
    Mateusz Rymuszka \and
    Piotr Wolski
}
\maketitle

\section{Założenia}

Dana jest klasa systemów dynamicznych spełniających następujące warunki:

\begin{enumerate}
    \item Prawo inercji.
    \item Niedeterminizm.
    \item W języku kwerend występują akcje złożone (zbiory co najwyżej $k$ akcji atomowych), w~języku akcji jedynie akcje atomowe.
    \item Pełna informacja o~wszystkich akcjach atomowych i~wszystkich ich skutkach bezpośrednich.
    \item Z~każdą akcją atomową związany jest jej warunek wstępny (ew. \textsc{True}) i~końcowy (efekt akcji).
    \item Wykonywane są jedynie akcje bezkonfliktowe (żadne dwie akcje składowe nie~mogą mieć wspólnych zmiennych, na~które w~żadnym stanie mają wpływ).
    \item Wynikiem akcji złożonej jest suma skutków wszystkich składowych akcji bezkonfliktowych.
    \item Akcje mogą być niewykonalne w~pewnych stanach; jeśli akcja jest niewykonywalna, to~każda akcja ją zawierająca też jest niewykonalna.
    \item Dopuszczalny jest opis częściowy zarówno stanu początkowego, jak i~pewnych stanów wynikających z~wykonań sekwencji akcji.
\end{enumerate}

\begin{defn}
    \textbf{Programem $P$ działań} jest ciąg $P = (A_1, \dots,A_n)$ akcji złożonych.
\end{defn}

\begin{defn}
    Program $P = (A_1, \dots, A_n)$ jest \textbf{realizowalny}, jeśli wszystkie akcje złożone $A_i$ są~wykonalne.
\end{defn}

Celem projektu jest opracowanie i~zaimplementowanie języka akcji dla~specyfikacji podanej klasy systemów dynamicznych oraz~odpowiadającego mu języka kwerend zapewniającego uzyskanie odpowiedzi na następujące zapytania:

\begin{enumerate}
    \item Czy podany program $P$ działań jest możliwy do~realizacji zawsze/kiedykolwiek ze stanu początkowego?
    \item Czy wykonanie programu $P$ w~stanie początkowym działań prowadzi zawsze/kiedykolwiek do~osiągnięcia celu~$\gamma$?
    \item Czy cel~$\gamma$ jest osiągalny ze~stanu początkowego?
\end{enumerate}

\section{Język akcji}

Scenariusze dla~zadanej klasy systemów dynamicznych będą reprezentowane za~pomocą języka akcji wzorowanego na~językach z~rodziny $\mathcal{AR}$.
Zgodnie z~założeniami projektu, w~języku akcji będą rozważane jedynie następstwa akcji atomowych.

\subsection{Składnia języka}

\begin{defn}
    \textbf{Sygnaturą} języka jest uporządkowana para zbiorów $\Upsilon = (\mathcal{F}, \mathcal{A}c)$, gdzie $\mathcal{F}$ to~zbiór fluentów, a $ \mathcal{A}c$ to~zbiór akcji.
    Dodatkowo definiuje~się zbiór \textbf{fluentów inercjalnych} $\mathcal{F}_I \subseteq \mathcal{F}$.
\end{defn}

Sygnatura języka określa zarówno logiczne zmienne określające stan świata (fluenty), jak i~możliwe metody manipulacji tym światem (akcje).

\begin{defn}
    \textbf{Literałem} $\overline{f}$ będziemy nazywać jedno z~wyrażeń
    $$ \overline{f} ::= f\ |\ \neg f $$
    gdzie $f \in \mathcal{F}$.
\end{defn}

\begin{defn}
    \textbf{Formuła} nad~zbiorem fluentów~$\mathcal{F}$ to~dowolny predykat zbudowany z~fluentów: 
    $$ \alpha ::= f\ |\ \neg\alpha\ |\ \alpha_1 \land \alpha_2\ |\ \alpha_1 \lor \alpha_2\ |\ \alpha_1 \rightarrow \alpha_2\ |\ \alpha_1 \equiv \alpha_2 $$
    Ponadto definiujemy dwie specjalne formuły:
    
    \begin{itemize}
    \item $\top\colon$ prawda,
    
    \item $\bot\colon$ fałsz.
    \end{itemize}
    
    Zbiór wszystkich formuł nad~zbiorem fluentów $\mathcal{F}$ oznaczamy jako $Forms(\mathcal{F})$.
\end{defn}

Proponowany język definiuje następujące siedem typów wyrażeń:

\begin{defn}
    \textbf{Zdaniem wartości początkowej} nazywamy zdanie postaci
    $$ \Initially \alpha $$
    gdzie $\alpha \in Forms(\mathcal{F})$.
\end{defn}

Zdanie wartości początkowej będzie służyło do~określenia początkowego stanu fluentów.
Zgodnie z~założeniami, w~systemie nie~jest wymagane wyspecyfikowanie wartości początkowej wszystkich fluentów.

\begin{defn}
    \textbf{Zdaniem wartości} nazywamy zdanie postaci
    $$ \alpha \After A $$
    gdzie $\alpha \in Forms(\mathcal{F})$ oraz $A \in \mathcal{A}c$.
\end{defn}

\begin{defn}
    \textbf{Zdaniem obserwacji} nazywamy zdanie postaci
    $$ \Observable \alpha \After A $$
    gdzie $\alpha \in Forms(\mathcal{F})$ oraz $A \in \mathcal{A}c$.
\end{defn}

Zdanie wartości mówi, że warunek~$\alpha$ musi zachodzić po~wykonaniu akcji~$A$, z~kolei zdanie obserwacji określa, że~warunek ten może być spełniony, ale niekoniecznie musi.

\begin{defn}
    \textbf{Zdaniem efektu} nazywamy zdanie postaci
    $$A \Causes \alpha \If \pi$$
    gdzie $\alpha, \pi \in Forms(\mathcal{F})$ oraz $A \in \mathcal{A}c$.

    W~szczególności:
    \begin{itemize}
        \item jeśli $\pi = \top$, to piszemy, że $A \Causes \alpha$,
        \item jeśli $\alpha = \bot$, to piszemy, że $\Impossible A \If \pi$.
    \end{itemize}
\end{defn}

Zdania efektu określają następstwa akcji, gdy spełniony jest warunek~$\alpha$, zwany również \textbf{warunkiem wstępnym} akcji.
Szczególną wagę ma~zdanie $\Impossible$, które może uniemożliwić wykonanie danej akcji atomowej pod~warunkiem spełnienia warunku wstępnego.

\begin{defn}
    \textbf{Zdaniem uwolnienia fluentu} nazywamy zdanie postaci
    $$ A \Releases f \If \pi $$
    gdzie $\pi \in Forms(\mathcal{F})$, $f \in \mathcal{F}_I$ oraz $A \in \mathcal{A}c$.

    W~szczególności, jeśli $\pi = \top$, to piszemy $A \Releases f$.
\end{defn}

Zdanie to~należy interpretować tak, że wykonanie akcji~$A$ w~każdym stanie może, ale nie~musi zmienić wartości fluentu inercjalnego~$f$.
Innymi słowy, może ono służyć do~modelowania akcji losowych.

\begin{defn}
    \textbf{Zdaniem ograniczenia} nazywamy zdanie postaci
    $$ \Always \alpha $$
    gdzie $\alpha \in Forms(\mathcal{F})$.
\end{defn}

Ograniczenia stosowane~są do~wyeliminowania pewnych niepożądanych stanów systemu.
Wyrażenie $\alpha$ zwane jest w~szczególności \textbf{warunkiem integralności}.

\begin{defn}
    \textbf{Zdaniem specyfikacji fluentu} nazywamy zdanie postaci
    $$ \Noninertial f $$
    gdzie $f \notin \mathcal{F}_I$. 
\end{defn}

Powyższe zdanie można traktować jak deklarację istnienia nieinercjalnego fluentu w~danej sygnaturze.

\begin{defn}
    Domeną akcji $\mathcal{D}$ nazywamy skończony zbiór zdań zdefiniowanych powyżej.
\end{defn}

\subsection{Semantyka języka}

\begin{defn}
    Stanem nazywamy funkcję $\sigma : \mathcal{F} \rightarrow \left\{0,1\right\}$.

    Dodatkowo, przez $\sigma^{*} : Forms(\mathcal{F}) \rightarrow \left\{0,1\right\}$ określamy rozszerzenie tej funkcji, zdefiniowane dla wszystkich formuł nad zbiorem fluentów:
    gdzie
    \begin{align*}
        \sigma^{*}\left(f\right) &= \sigma\left(f\right) \text{ dla } f \in \mathcal{F} \\
        \sigma^{*}\left(\neg \alpha\right) &= 1 - \sigma^{*}\left(\alpha\right) \\
        \sigma^{*}\left(\alpha \land \beta\right) &= \min \left(\sigma^{*}\left(\alpha\right),\sigma^{*}\left(\beta\right)\right) \\
        \sigma^{*}\left(\alpha \lor \beta\right) &= \max \left(\sigma^{*}\left(\alpha\right),\sigma^{*}\left(\beta\right)\right) \\
        \sigma^{*}\left(\alpha \rightarrow \beta\right) &= \sigma^{*}\left(\neg \alpha \lor \beta\right) \\
        \sigma^{*}\left(\alpha \equiv \beta\right) &= \sigma^{*}\left(\left(\alpha \rightarrow \beta\right)\land\left(\beta \rightarrow \alpha\right)\right)
    \end{align*}
    Dla każdej formuły $\alpha \in Forms\left(\mathcal{F}\right)$, gdy $\sigma^*\left(\alpha\right) = 1$, piszemy $\sigma \models \alpha$.
\end{defn}

Funkcja stanu definiuje w~sposób jednoznaczny wartość wszystkich fluentów.

\begin{defn}
    \textbf{Strukturą} dla języka $\mathcal{L}$ nazywamy trójkę
    $$\mathcal{S} = \left(\Sigma, \sigma_0, Res\right)$$
    gdzie:
    \begin{enumerate}
        \item $\Sigma$ to zbiór stanów,
        \item $\sigma_0 \in \Sigma$ to stan początkowy,
        \item $Res : 2^{\mathcal{A}c} \times \Sigma \rightarrow 2^\Sigma$ to funkcja przejścia.
    \end{enumerate}
\end{defn}

Docelowo, dla~każdej akcji złożonej $A' \in 2^{\mathcal{A}c}$ i dla~każdego stanu $\sigma \in \Sigma$, $Res$ przyporządkuje zbiór stanów możliwych do osiągnięcia przez wykonanie akcji~$A'$ w~stanie $\sigma$.

Zauważmy, że~docelowo w~systemie może~być rozważane wiele struktur dla~jednej domeny akcji, różniących~się stanem początkowym, z~racji niepełnej wiedzy o~początkowej wartości fluentów.

Funkcję przejścia możemy rozszerzyć na~pojęcie programu w~następujący sposób:

\begin{defn}
    Niech $\mathcal{S} = (\Sigma, \sigma_0, Res)$ będzie strukturą.
    Niech $Res^* : (2^{\mathcal{A}c})^* \times \Sigma \to 2^\Sigma$ będzie funkcją taką, że:
    \begin{itemize}
        \item $Res^*(\varepsilon, \sigma) = \{ \sigma \}$,
        \item $Res^*((A_1), \sigma) = Res(A_1, \sigma)$,
        \item $Res^*((A_1,A_2,\dots,A_n,A_{n+1}), \sigma) = \bigcup_{\sigma' \in Res^*((A_1,A_2,\dots,A_n), \sigma)} Res(A_{n+1}, \sigma')$.
    \end{itemize}
\end{defn}

Wtedy pojedynczą ścieżkę wykonania programu możemy odtworzyć za~pomocą funkcji $\Psi_{\mathcal{S}}$ zdefiniowaną następująco:

\begin{defn}
    Niech $\mathcal{D}$ będzie domeną akcji. Dla struktury $S = (\Sigma,\sigma_0,Res)$, definiujemy funkcję częściową 
    $\Psi_{\mathcal{S}} : (2^{\mathcal{A}c})^* \times \Sigma \rightarrow \Sigma$ w następujący sposób:
    \begin{itemize}
    \item $\Psi_{\mathcal{S}}(\varepsilon,\sigma) = \sigma$
    \item jeśli dla $n \geq 1, \Psi_{\mathcal{S}}((A_1,\dots,A_n),\sigma)$ jest zdefiniowane i $Res^*((A_1,\dots,A_n,A_{n+1}),\sigma) \neq \emptyset$, to~$\Psi_{\mathcal{S}}((A_1,\dots,A_n,A_{n+1}),\sigma) \in Res(A_{n+1},\Psi_{\mathcal{S}}((A_1,\dots,A_n),\sigma)).$
    \end{itemize}
\end{defn}

\subsubsection{Akcje złożone}

Zanim zdefiniujemy model dla~domen akcji rozważanego języka, należy określić, na~czym polega wykonanie akcji złożonej z~wielu akcji atomowych.
Zauważmy, że w~domenie akcji mogą znajdować~się zdania, których efekty będą ze~sobą wzajemnie sprzeczne, tj. dla pewnego zbioru akcji atomowych i~pewnego stanu nie~będzie możliwe spełnienie ich warunków końcowych.
Do~sformalizowania tej intuicji posłużymy~się dysjunkcyjną postacią normalną formuł rachunku zdań.

\begin{defn}
    Niech $\alpha$ będzie dowolną formułą nad zbiorem fluentów $\mathcal{F}$.
    Mówimy, że $\alpha$ jest w~\textbf{dysjunkcyjnej postaci normalnej (DNF)} wtedy i tylko wtedy, gdy 
    $$\alpha = \bigvee_{i=1}^{n} \bigwedge_{j=1}^{k(i)} \lambda_{ij}$$
    gdzie $\lambda_{ij}$ jest literałem.
\end{defn}
    
\begin{theorem}
    Każdą formułę $\alpha$ można przekształcić do formuły $\alpha'$, która jest w postaci DNF.
\end{theorem}

Formułę $\alpha$ przekształconą do postaci DNF oznaczamy jako $DNF(\alpha)$.

\begin{defn}
    Niech $DNF(\alpha) = \gamma_1 \vee \dots \vee \gamma_k$.
    Każdą formułę $\gamma_i$ nazywamy \textbf{komponentem} formuły $\alpha$.
\end{defn}

Po~zdefiniowaniu DNF przystąpmy do~określenia, jakie akcje będą~rozważane za~konfliktowe.

\begin{defn}
    \label{defn:conflict}
    Niech $\mathcal{D}$ będzie domeną akcji, $\sigma$ będzie pewnym stanem oraz niech $A, B \in \mathcal{A}c$ będą akcjami elementarnymi. Akcje $A, B$ są~\textbf{konfliktowe} wtedy i tylko wtedy, gdy przynajmniej jeden z~następujących warunków jest prawdziwy:
    \begin{itemize}
        \item W~$D$ znajdują~się dwa zdania postaci:
        \begin{align*}
            A & \Causes \alpha \If \pi_1 \\
            B & \Causes \beta \If \pi_2
        \end{align*}
        takie, że:
        \begin{itemize}
            \item $\sigma \models \pi_1 \land \pi_2$
            \item Jakiś komponent z $\alpha$ jest sprzeczny z~jakimś komponentem z~$\beta$.
        \end{itemize}
        \item W~$D$ znajdują~się dwa zdania postaci:
        \begin{align*}
            A & \Causes \alpha \If \pi_1 \\
            B & \Releases f \If \pi_2
        \end{align*}
        takie, że:
        \begin{itemize}
            \item $\sigma \models \pi_1 \land \pi_2$
            \item w~jednym z~komponentów $\alpha$ występuje $f$.
        \end{itemize}
        \item W~$D$ znajdują~się dwa zdania postaci:
        \begin{align*}
            A & \Releases f \If \pi_1 \\
            B & \Releases f \If \pi_2
        \end{align*}
        takie, że $\sigma \models \pi_1 \land \pi_2$.
    \end{itemize}
\end{defn}

Zauważmy, że~w~pierwszym przypadku warunek o~sprzeczności komponentów z~$\alpha$ i~$\beta$ nie~jest równoważny sprzeczności formuły $\alpha \land \beta$.

\begin{defn}
    \label{defn:potentially-executable}
    Niech $\mathcal{D}$ to domena akcji, niech $\sigma$ to pewien stan. Akcja złożona $\{ A_1, \dots A_k \}$, gdzie $A_i$ to akcje elementarne, jest \textbf{potencjalnie wykonywalna} w $\sigma$ wtedy i tylko wtedy gdy nie istnieje $B \subseteq A_1 \cup \dots \cup A_k$ takie, że~$\mathcal{D}$ zawiera zdania:
    \begin{align*}
        & B \Causes \alpha_1 \If \pi_1 \\
        & \cdots \\
        & B \Causes \alpha_k \If \pi_k
    \end{align*}
    takie, że:
    \begin{itemize}
        \item $\sigma \models \pi_1 \land \dots \land \pi_k$
        \item $\alpha_1 \land \dots \land \alpha_k$ jest sprzeczne.
    \end{itemize}
\end{defn}

\begin{defn}
    Niech $\mathcal{D}$ będzie domeną akcji, niech $\sigma$ będzie stanem oraz niech $A \subseteq \mathcal{A}c$.
    \textbf{Dekompozycją} $A$ w $\sigma$ będziemy nazywać taki zbiór $\Delta(A, \sigma) = \{ E_1, E_2, \dots, E_k \}$ akcji elementarnych w $D$, że:
    \begin{enumerate}
        \item $\bigcup_{i = 1}^{k} E_i \subseteq A$,
        \item $\Delta(A, \sigma)$ jest zbiorem niekonfliktujących akcji potencjalnie wykonywalnych w  $\sigma$,
        \item Nie ma żadnego zbioru $\{ E'_1, \dots, E'_k, E'_{k+1}, \dots, E'_m \}$ akcji spełniającego powyższe warunki takiego, że $E_i \subset E_i^\prime$ dla pewnego $i = 1, \dots, k$.
    \end{enumerate}
\end{defn}

Innymi słowy, dekompozycja akcji złożonej $A$ określa maksymalne ze~względu na~inkluzję potencjalnie wykonywane podakcje $A$.
Zauważmy, że na~mocy definicji \ref{defn:conflict} i~\ref{defn:potentially-executable} możemy wspomóc~się dysjunkcyjną postacią normalną przy~wyznaczaniu niekonfliktujących akcji.
    
\begin{defn}
    Niech $\Delta(A, \sigma) = \{ E_1, E_2, \dots, E_k \}$ będzie dekompozycją $A$ w $\sigma$. Każdą akcję $E_i$ nazywamy \textbf{komponentem wykonania} $A$ w~stanie $\sigma$.
\end{defn}
    
\begin{defn}
    Niech $\mathcal{D}$ będzie domeną akcji, niech $\sigma, \sigma^\prime$ będą stanami, niech $A \subseteq \mathcal{A}c$ będzie akcją i niech $\Delta(A, \sigma) = \{ E_1, E_2, \dots, E_k \}$ będzie dekompozycją $A$ w~$\sigma$.
    Mówimy, że $\sigma^\prime$ jest \textbf{potencjalnym wynikiem wykonania} $A$ w $\sigma$ wtedy i~tylko wtedy gdy dla każdego $i = 1, \dots, k$ i~dla~każdego zdania postaci: 
    $$ (E_i \Causes \alpha \If \pi) \in D $$
    zachodzi $\sigma \models \pi \Rightarrow \sigma^\prime \models \alpha $.
\end{defn} 
    
Oznacza to, że po wykonaniu $A$ w stanie $\sigma$ stan wynikowy to taki, który spełnia wszystkie efekty każdego komponentu $E_i$ z~danej dekompozycji.
Wybór dekompozycji jest jednym z~elementów niedeterminizmu w~rozważanym systemie.

Zauważmy ponadto, że~takie zdefiniowanie potencjalnego wyniku wykonania akcji zapewnia, że~jeśli akcja złożona zawiera akcję atomową niewykonalną w~$\sigma$, to~cała akcja złożona również nie~będzie wykonalna.

\subsubsection{Model domeny akcji}

Niech $\mathcal{D}$ będzie domeną akcji, zaś $\mathcal{S} = \left(\Sigma, \sigma_0, Res\right)$ pewną strukturą języka.

\begin{defn}
    Fluent $f \in \mathcal{F}$ jest inercjalny w~$\mathcal{S}$ wtedy i~tylko wtedy, gdy $\left(\Noninertial f\right) \notin \mathcal{D}$.
\end{defn}

\begin{defn}
    $\sigma : \mathcal{F} \rightarrow \left\{0,1\right\}$ jest \textbf{stanem} $\mathcal{S}$ względem~$\mathcal{D}$ wtedy i~tylko wtedy, gdy dla~każdego zdania ograniczenia $\left(\Always \alpha\right) \in \mathcal{D}$ zachodzi $\sigma \models \alpha$.
\end{defn}

Powyższa definicja stanu względem~$\mathcal{D}$ zapewni spełnienie warunków integralności w~każdym modelu domeny~$\mathcal{D}$.

\begin{defn}
    Zdanie wartości $(\alpha \After A)$ jest \textbf{prawdziwe} w~$\mathcal{S}$ wtedy i~tylko wtedy, gdy
    $$\Psi_{\mathcal{S}}(\{ A \}, \sigma_0) \models \alpha$$
    dla~każdej funkcji $\Psi_{\mathcal{S}}$.
\end{defn}

\begin{defn}
    Zdanie obserwacji $(\Observable \alpha \After A)$ jest \textbf{prawdziwe} w~$\mathcal{D}$ wtedy i~tylko wtedy, gdy
    $$\Psi_{\mathcal{S}}(\{ A \}, \sigma_0) \models \alpha$$
    dla~pewnej funkcji $\Psi_{\mathcal{S}}$.
\end{defn}

Aby móc zdefiniować warunki stawiane funkcji przejścia w~modelu, wprowadzimy kilka pojęć pomocniczych.

\begin{defn}
    Zdefiniujmy pomocniczą funkcję $Res_0: 2^{\mathcal{A}c} \times \Sigma \rightarrow 2^\Sigma$.

    Niech dany będzie stan $\sigma \in \Sigma$ i~akcja złożona $A \subseteq \mathcal{A}c$.
    Niech $\Delta_1(A, \sigma), \dots, \Delta_n(A, \sigma)$ będą dekompozycjami akcji~$A$ względem stanu~$\sigma$.
    Wówczas stan $\sigma' \in Res_0(\sigma, A)$ wtedy i~tylko wtedy, gdy $\sigma'$ jest potencjalnym wynikiem wykonania $A$ dla~pewnej dekompozycji $\Delta_i(A, \sigma)$.
\end{defn}

Funkcja $Res_0$ określa możliwe zmiany w~stanie~$\sigma$ spowodowane przez~zdania $\Causes$.

\begin{defn}
    Dla~każdych stanów $\sigma, \sigma' \in \Sigma$ i~dla~każdej akcji $A \in \mathcal{A}c$, zbiorem $New(A, \sigma, \sigma')$ będziemy nazywać zbiór wszystkich literałów $\overline{f}$ takich, że $\sigma' \models f$ oraz:
    \begin{itemize}
        \item $f$ jest inercjalny oraz $\sigma(f) \neq \sigma'(f)$, lub
        \item w~$\mathcal{D}$ istnieje zdanie $(A \Releases f \If \pi)$ oraz~$\sigma \models \pi$.
    \end{itemize}
\end{defn}

Zbiory $New$ określają, które fluenty inercjalne zmieniły~się pomiędzy dwoma stanami, a~ponadto uwzględniają możliwość dowolnych fluktuacji fluentów uwolnionych.
Jest to drugi aspekt niedeterministyczny w~rozważanym systemie dynamicznym.

\begin{defn}
    Niech $\mathcal{D}$ będzie domeną akcji w $\mathcal{A}c$. Struktura $\mathcal{S} = (\Sigma,\sigma_0,Res)$ jest \textbf{modelem $\mathcal{D}$} wtw:
    \begin{enumerate}
    \item $\Sigma$ jest zbiorem wszystkich stanów w $\mathcal{D}$ (tj. zbiorem wszystkich stanów satysfakcjonujących wszystkie ograniczenia z $\mathcal{D}$),
    \item każde zdanie wartości i każde zdanie obserwacji jest prawdziwe w $\mathcal{S}$,
    \item dla każdej akcji $A \subseteq \mathcal{A}c$ i dla każdego stanu $\sigma \in \Sigma$, $Res(A,\sigma) \subseteq Res_0(A,\sigma)$ jest zbiorem wszystkich stanów $\sigma'$, dla~których zbiory $New(A,\sigma,\sigma')$ są~minimalne ze~względu na~relację inkluzji.
    \end{enumerate}
\end{defn}

Tak zdefiniowany model łączy niedeterminizm związany z~wyborem dekompozycji akcji złożonych (funkcja $Res_0$) i~niedeterminizm związany z~fluentami uwolnionymi (zbiór $New$).

\section{Język kwerend}

W~języku kwerend formułowane będą zapytania stawiane systemowi dynamicznemu, którego zadaniem jest podanie odpowiedzi zgodnych z~daną domeną akcji.

\subsection{Składnia języka}
    Język kwerend wyróżnia ich następujące rodzaje:

    \begin{defn}
        \textbf{Egzystencjalną kwerendą wartości} nazywamy zdanie
        $$ \Possibly \gamma \After A_1, \dots, A_n $$
        gdzie $\gamma \in Forms(f)$, $A_1, \dots, A_n \subseteq \mathcal{A}c$.
    \end{defn}

    \begin{defn}
        \textbf{Ogólną kwerendą wartości} nazywamy zdanie
        $$ \Necessary \gamma \After A_1, \dots, A_n $$
        gdzie $\gamma \in Forms(f)$, $A_1, \dots, A_n \subseteq \mathcal{A}c$.
    \end{defn}

    Kwerenda egzystencjalna określa, czy wykonanie programu $(A_1, \dots, A_n)$ ze~stanu początkowego prowadzi kiedykolwiek do~osiągnięcia celu~$\gamma$; z~kolei ogólna określa, czy wykonanie programu $(A_1, \dots, A_n)$ ze~stanu początkowego zawsze prowadzi do~osiągnięciu celu~$\gamma$.

    \begin{defn}
        \textbf{Egzystencjalną kwerendą wykonywalności} nazywamy zdanie
        $$ \Executable \Sometimes A_1, \dots, A_n $$
        gdzie $A_1, \dots, A_n \subseteq \mathcal{A}c$.
    \end{defn}

    \begin{defn}
        \textbf{Ogólną kwerendą wykonywalności} nazywamy zdanie
        $$ \Executable \Always A_1, \dots, A_n $$
        gdzie $A_1, \dots, A_n \subseteq \mathcal{A}c$.
    \end{defn}

    Kwerendy wykonywalności mają na~celu odpowiedzieć na~pytanie, czy~dany program jest kiedykolwiek (egzystencjalna) lub~zawsze (ogólna) wykonywalny ze~stanu początkowego.

    \begin{defn}
        \textbf{Kwerendą osiągalności} nazywamy zdanie
        $$ \Accessible \gamma $$
        gdzie $\gamma \in Forms(f)$.
    \end{defn}
    
    Celem tej kwerendy jest określenie, czy~istnieje taki program $(A_1, \dots, A_n)$, $n \geq 0$, wykonywalny ze~stanu początkowego, po~którego wykonaniu osiągnięty zostanie cel~$\gamma$.

\subsection{Semantyka języka}

\begin{defn}
    Niech $\mathcal{D}$ będzie domeną akcji, a $\mathcal{Q}$ kwerendą. 
    Mówimy, że $\mathcal{Q}$ jest \textbf{konsekwencją} $\mathcal{D}$ (ozn. $\mathcal{D} \consequence \mathcal{Q}$), jeżeli:

    \begin{enumerate}
        \item w przypadku gdy $\mathcal{Q}$ jest kwerendą wartości postaci:
        \begin{itemize}
            \item $\Possibly \gamma \After A_1, \dots, A_n$ -- wówczas $\mathcal{D} \consequence \mathcal{Q}$ wtedy i tylko wtedy, gdy dla~każdego modelu % TODO: do uzgodnienia
            $\mathcal{S} = (\Sigma,\sigma_0,Res)$ domeny akcji $\mathcal{D}$ istnieje funkcja $\Psi_\mathcal{S}$ taka, że zachodzi $$\Psi_S((A_1,...,A_n),\sigma_{0}) \models \gamma$$
            \item $\Necessary \gamma \After A_1, \dots, A_n$ -- wówczas $\mathcal{D} \consequence \mathcal{Q}$ wtedy i tylko wtedy, gdy dla każdego modelu % TODO: do uzgodnienia
            $\mathcal{S} = (\Sigma,\sigma_0,Res)$ domeny akcji $\mathcal{D}$ i~dla~każdej funkcji $\Psi_\mathcal{S}$ zachodzi $$\Psi_\mathcal{S}((A_1,...,A_n),\sigma_{0}) \models \gamma$$
        \end{itemize}
        \item w przypadku gdy $\mathcal{Q}$ jest kwerendą wykonywalności postaci:
        \begin{itemize}
            \item $\Executable \Sometimes A_1, \dots, A_n$ -- wówczas $\mathcal{D} \consequence \mathcal{Q}$ wtedy i tylko wtedy, gdy dla~każdego modelu % TODO: do uzgodnienia
            $\mathcal{S} = (\Sigma,\sigma_0,Res)$ domeny akcji $\mathcal{D}$ takiego, że istnieje % TODO: do uzgodnienia
            funkcja $\Psi_\mathcal{S}$ taka, że
            $$\Psi_\mathcal{S}((A_1,...,A_n),\sigma_{0})$$
            jest zdefiniowane,
            \item $\Executable \Always A_1, \dots, A_n$ -- wówczas $\mathcal{D} \consequence \mathcal{Q}$ wtedy i tylko wtedy, gdy dla każdego modelu % TODO: do uzgodnienia
            $\mathcal{S} = (\Sigma,\sigma_0,Res)$ domeny akcji $\mathcal{D}$ i~dla~każdej funkcji $\Psi_\mathcal{S}$
            $$\Psi_\mathcal{S}((A_1,...,A_n),\sigma_{0})$$
            jest zdefiniowane,
        \end{itemize}
        \item w przypadku gdy $\mathcal{Q}$ jest kwerendą osiągalności postaci $\Accessible \gamma$ -- wówczas $\mathcal{D} \consequence \mathcal{Q}$ wtedy i tylko wtedy, gdy dla każdego modelu $\mathcal{S} = (\Sigma,\sigma_0,Res)$ domeny akcji $\mathcal{D}$, dla każdej funkcji $\Psi_\mathcal{S}$ istnieje ciąg akcji $(A_1,...,A_n), \; n \geq 0$ taki, że:
        \begin{itemize}
            \item $\Psi_\mathcal{S}((A_1,...,A_n),\sigma_{0})$ jest zdefiniowane,
            \item $\Psi_\mathcal{S}((A_1,...,A_n),\sigma_{0}) \models \gamma$.
        \end{itemize}
    \end{enumerate}
\end{defn}

\section{Przykłady}

\subsection{Producent i~konsument}

Załóżmy, że~dany jest producent i~konsument.
Producent może wstawić do~bufora dokładnie jeden produkt przy~pomocy akcji \textsc{Put}.
Konsument może pobrać z~bufora produkt z~użyciem akcji \textsc{Get},
oraz~pozbyć~się go z~użyciem akcji \textsc{Consume}.
Konsument nie~może skonsumować produktu, jeśli wcześniej go nie~pobrał.
Na~początku bufor jest pusty.

Reprezentacja tego scenariusza w~postaci domeny akcji $\mathcal{D}$ wygląda następująco:
\begin{align*}
    & \textsc{Put} \Causes \neg \textit{bufferEmpty} \If \textit{bufferEmpty} \\
    & \textsc{Get} \Causes \textit{bufferEmpty} \land \textit{hasItem} \If \neg \textit{bufferEmpty} \land \neg \textit{hasItem} \\
    & \textsc{Consume} \Causes \neg \textit{hasItem} \\
    & \Impossible \textsc{Consume} \If \neg \textit{hasItem} \\
    & \Initially \textit{bufferEmpty}
\end{align*}

Rozważmy po~kolei wykonanie wszystkich możliwych akcji złożonych w~każdym stanie. Możliwe~są cztery stany:

\begin{enumerate}
    \item $\sigma_0 = \{ \textit{bufferEmpty}, \textit{hasItem} \}$,
    \item $\sigma_1 = \{ \neg \textit{bufferEmpty}, \textit{hasItem} \}$,
    \item $\sigma_2 = \{ \textit{bufferEmpty}, \neg \textit{hasItem} \}$,
    \item $\sigma_3 = \{ \neg \textit{bufferEmpty}, \neg \textit{hasItem} \}$.
\end{enumerate}

Potencjalnymi stanami początkowymi są $\sigma_0$ oraz $\sigma_2$.

\begin{itemize}
    \item Dla~stanu $\sigma_0$ mamy
    \begin{align*}
        Res_0(\sigma_0, \{ \textsc{Put} \}) &= \{ \sigma_1 \} = Res(\sigma_0, \{ \textsc{Put} \}) \\
        Res_0(\sigma_0, \{ \textsc{Get} \}) &= \{ \sigma_0 \} = Res(\sigma_0, \{ \textsc{Get} \}) \\
        Res_0(\sigma_0, \{ \textsc{Consume} \}) &= \{ \sigma_2 \} = Res(\sigma_0, \{ \textsc{Consume} \}) \\
        Res_0(\sigma_0, \{ \textsc{Put, Get} \}) &= \{ \sigma_1 \} = Res(\sigma_0, \{ \textsc{Put, Get} \}) \\
        Res_0(\sigma_0, \{ \textsc{Put, Consume} \}) &= \{ \sigma_3 \} = Res(\sigma_0, \{ \textsc{Put, Consume} \}) \\
        Res_0(\sigma_0, \{ \textsc{Get, Consume} \}) &= \{ \sigma_2 \} = Res(\sigma_0, \{ \textsc{Get, Consume} \}) \\
        Res_0(\sigma_0, \{ \textsc{Put, Get, Consume} \}) &= \{ \sigma_3 \} = Res(\sigma_0, \{ \textsc{Put, Get, Consume} \})
    \end{align*}
    
    \item Dla~stanu $\sigma_1$ mamy
    \begin{align*}
        Res_0(\sigma_1, \{ \textsc{Put} \}) &= \{ \sigma_1 \} = Res(\sigma_1, \{ \textsc{Put} \}) \\
        Res_0(\sigma_1, \{ \textsc{Get} \}) &= \{ \sigma_1 \} = Res(\sigma_1, \{ \textsc{Get} \}) \\
        Res_0(\sigma_1, \{ \textsc{Consume} \}) &= \{ \sigma_3 \} = Res(\sigma_1, \{ \textsc{Consume} \}) \\
        Res_0(\sigma_1, \{ \textsc{Put, Get} \}) &= \{ \sigma_1 \} = Res(\sigma_1, \{ \textsc{Put, Get} \}) \\
        Res_0(\sigma_1, \{ \textsc{Put, Consume} \}) &= \{ \sigma_3 \} = Res(\sigma_1, \{ \textsc{Put, Consume} \}) \\
        Res_0(\sigma_1, \{ \textsc{Get, Consume} \}) &= \{ \sigma_3 \} = Res(\sigma_1, \{ \textsc{Get, Consume} \}) \\
        Res_0(\sigma_1, \{ \textsc{Put, Get, Consume} \}) &= \{ \sigma_3 \} = Res(\sigma_1, \{ \textsc{Put, Get, Consume} \})
    \end{align*}
    
    \item Dla~stanu $\sigma_2$ mamy
    \begin{align*}
        Res_0(\sigma_2, \{ \textsc{Put} \}) &= \{ \sigma_3 \} = Res(\sigma_2, \{ \textsc{Put} \}) \\
        Res_0(\sigma_2, \{ \textsc{Get} \}) &= \{ \sigma_2 \} = Res(\sigma_2, \{ \textsc{Get} \}) \\
        Res_0(\sigma_2, \{ \textsc{Consume} \}) &= \emptyset = Res(\sigma_2, \{ \textsc{Consume} \}) \\
        Res_0(\sigma_2, \{ \textsc{Put, Get} \}) &= \{ \sigma_3 \} = Res(\sigma_2, \{ \textsc{Put, Get} \}) \\
        Res_0(\sigma_2, \{ \textsc{Put, Consume} \}) &= \emptyset = Res(\sigma_2, \{ \textsc{Put, Consume} \}) \\
        Res_0(\sigma_2, \{ \textsc{Get, Consume} \}) &= \emptyset = Res(\sigma_2, \{ \textsc{Get, Consume} \}) \\
        Res_0(\sigma_2, \{ \textsc{Put, Get, Consume} \}) &= \emptyset = Res(\sigma_2, \{ \textsc{Put, Get, Consume} \})
    \end{align*}
    Występowanie zbiorów pustych w~akcjach złożonych zawierających akcję atomową \textsc{Consume} zapewnia definicja \ref{defn:potentially-executable} -- nie~istnieje żaden potencjalny wynik wykonania tych akcji.

    \item Dla~stanu $\sigma_3$ mamy
    \begin{align*}
        Res_0(\sigma_3, \{ \textsc{Put} \}) &= \{ \sigma_3 \} = Res(\sigma_3, \{ \textsc{Put} \}) \\
        Res_0(\sigma_3, \{ \textsc{Get} \}) &= \{ \sigma_0 \} = Res(\sigma_3, \{ \textsc{Get} \}) \\
        Res_0(\sigma_3, \{ \textsc{Consume} \}) &= \emptyset = Res(\sigma_3, \{ \textsc{Consume} \}) \\
        Res_0(\sigma_3, \{ \textsc{Put, Get} \}) &= \{ \sigma_0 \} = Res(\sigma_3, \{ \textsc{Put, Get} \}) \\
        Res_0(\sigma_3, \{ \textsc{Put, Consume} \}) &= \emptyset = Res(\sigma_3, \{ \textsc{Put, Consume} \}) \\
        Res_0(\sigma_3, \{ \textsc{Get, Consume} \}) &= \emptyset = Res(\sigma_3, \{ \textsc{Get, Consume} \}) \\
        Res_0(\sigma_3, \{ \textsc{Put, Get, Consume} \}) &= \emptyset = Res(\sigma_3, \{ \textsc{Put, Get, Consume} \})
    \end{align*}
\end{itemize}

Warto zauważyć, że w~tym przypadku funkcja przejścia jest całkowicie deterministyczna, więc istnieje dokładnie jedna funkcja $\Psi_\mathcal{S}$ (choć jest ona nadal częściowo zdefiniowana).
Graf funkcji przejścia dla~tego scenariusza znajduje~się na~rysunku~\ref{fig:producer-consumer-state}.

\begin{figure}[H]
    \centering
    \begin{tikzpicture}[->,node distance=9cm]
        \tikzstyle{state} = [shape=rectangle,draw]
        \tikzstyle{initial} = [fill=gray!25]
        \node [initial,state] (s0) {$\sigma_0 = \{ \textit{BE}, \textit{HI} \}$};
        \node [right of=s0,state] (s1) {$\sigma_1 = \{ \neg \textit{BE}, \textit{HI} \}$};
        \node [below=of s0,initial,state] (s2) {$\sigma_2 = \{ \textit{BE}, \neg \textit{HI} \}$};
        \node [below=of s1,state] (s3) {$\sigma_3 = \{ \neg \textit{BE}, \neg \textit{HI} \}$};

        \path (s0)  edge node [above] {$\{ \textsc{Put} \}, \{ \textsc{Put, Get} \}$} (s1)
                    edge [out=90, in=180, looseness=3] node [above left] {$\{ \textsc{Get} \}$} (s0)
                    edge node [left, align=right] {$\{ \textsc{Consume} \}$ \\ $\{ \textsc{Get, Consume} \}$} (s2)
                    edge node [below left, align=right] {$\{ \textsc{Put, Consume} \}$ \\ $\{ \textsc{Put, Get, Consume} \}$} (s3);
        \path (s1)  edge [out=90, in=0, looseness=3] node [above right, align=left]             {$\{ \textsc{Put} \}$ \\ $\{ \textsc{Get} \}$ \\ $\{ \textsc{Put, Get} \}$} (s1)
                    edge node [right, align=left] {$\{ \textsc{Consume} \}$ \\ $\{ \textsc{Put, Consume} \}$ \\ $\{ \textsc{Get, Consume} \}$ \\ $\{ \textsc{Put, Get, Consume} \}$} (s3);
        \path (s2)  edge node [below] {$\{ \textsc{Put} \}$, $\{ \textsc{Put, Get} \}$} (s3)
                    edge [out=270, in=180, looseness=3] node [below left] {$\{ \textsc{Get} \}$} (s2);

        \path (s3)  edge [out=270, in=0, looseness=3] node [below right, align=left] {$\{ \textsc{Put} \}$} (s3)
                    edge [bend right=20] node [above right, align=left] {$\{ \textsc{Get} \}$ \\ $\{ \textsc{Put, Get} \}$} (s0);
    \end{tikzpicture}
    \caption{
        Graf funkcji przejścia dla~scenariusza \emph{producent-konsument}.
        Kolorem szarym oznaczone potencjalne stany początkowe.
    }
    \label{fig:producer-consumer-state}
\end{figure}

Rozważmy teraz kilka przykładowych kwerend nad~$\mathcal{D}$.

\begin{enumerate}
    \item Niech dana będzie kwerenda
    $$ \Possibly \neg \textit{bufferEmpty} \land \neg \textit{hasItem} \After \{ \textsc{Put} \}, \{ \textsc{Get, Consume} \} $$
    Kwerenda ta \textbf{będzie} konsekwencją $\mathcal{D}$, ponieważ dla~modelu~$\mathcal{D}$, którego stanem początkowym jest~$\sigma_0$, program $(\{ \textsc{Put} \}, \{ \textsc{Get, Consume} \})$ prowadzi dla~wszystkich $\Psi_\mathcal{S}$ (tj. dla~jedynej istniejącej funkcji $\Psi_\mathcal{S}$) do~osiągnięcia zadanego celu $\textit{bufferEmpty} \land \neg \textit{hasItem}$.

    \item Niech dana będzie kwerenda
    $$ \Necessary \neg \textit{bufferEmpty} \land \neg \textit{hasItem} \After \{ \textsc{Put} \}, \{ \textsc{Get, Consume} \} $$
    Kwerenda ta \textbf{nie~będzie} konsekwencją $\mathcal{D}$, ponieważ w~modelu~$\mathcal{D}$, którego stanem początkowym jest~$\sigma_2$, program $(\{ \textsc{Put} \}, \{ \textsc{Get, Consume} \})$ nie~jest wykonywalny, więc odpowiednia wartość $\Psi_\mathcal{S}$ w~definicji konsekwencji nie~jest zdefiniowana.

    \item Analogicznie, kwerenda
    $$\Executable \Sometimes (\{ \textsc{Put} \}, \{ \textsc{Get, Consume} \})$$
    \textbf{jest} konsekwencją $\mathcal{D}$, podczas gdy
    $$\Executable \Always (\{ \textsc{Put} \}, \{ \textsc{Get, Consume} \})$$
    \textbf{nie jest} konsekwencją $\mathcal{D}$.

    \item Niech dana będzie kwerenda
    $$ \Accessible \textit{bufferEmpty} \lor \neg \textit{hasItem} $$
    Kwerenda ta \textbf{jest} konsekwencją $\mathcal{D}$, ponieważ w~szczególności program pusty prowadzi do~spełnienia celu $\textit{bufferEmpty} \lor \neg \textit{hasItem}$ (tj. w~każdym modelu stan początkowy spełnia zadany cel).
\end{enumerate}

\subsection{Kot Schrödingera}
Załóżmy, że dany jest pojemnik z kotem i trucizną, która w każdym momencie może zostać uwolniona. Na początku wiadomo, że kot żyje. Przy pomocy akcji \textsc{Peek} można zajrzeć do pojemnika, po czym może się okazać, że kot umarł, zaczął lub przestał mruczeć. Jeżeli kot żyje, wykonanie akcji \textsc{Pet} spowoduje, że kot zacznie mruczeć. Jeżeli kot nie żyje, niemożliwe jest żeby mruczał.

Powyższy scenariusz jest reprezentowany przez następującą domenę akcji $\mathcal{D}$:

\begin{align*}
    & \Initially \textit{alive} \\
    & \textsc{Peek} \Releases \textit{alive} \If \textit{alive} \\
    & \textsc{Peek} \Releases \textit{purring} \If \textit{alive} \\
    & \textsc{Pet} \Causes \textit{purring} \If  \textit{alive} \\
    & \Always \neg \textit{alive} \rightarrow \neg \textit{purring}
\end{align*}

Rozważmy po~kolei wykonanie wszystkich możliwych akcji złożonych w~każdym stanie. Możliwe~są trzy stany:

\begin{enumerate}
    \item $\sigma_0 = \{ \textit{alive}, \textit{purring} \}$,
    \item $\sigma_1 = \{ \textit{alive}, \neg \textit{purring} \}$,
    \item $\sigma_2 = \{ \neg \textit{alive}, \neg \textit{purring} \}$.
\end{enumerate}

Potencjalnymi stanami początkowymi są $\sigma_0$ oraz $\sigma_1$.


\begin{itemize}
    \item Dla~stanu $\sigma_0$ mamy
    \begin{align*}
        Res_0(\sigma_0, \{ \textsc{Peek} \}) &= \{ \sigma_0, \sigma_1, \sigma_2 \} \\
        New(\{ \textsc{Peek} \}, \sigma_0, \sigma_0) &=  \{ \textit{alive}, \textit{purring} \} \\
        New(\{ \textsc{Peek} \}, \sigma_0, \sigma_1) &=  \{ \textit{alive}, \neg \textit{purring} \} \\
        New(\{ \textsc{Peek} \}, \sigma_0, \sigma_2) &=  \{ \neg \textit{alive}, \neg \textit{purring} \} \\
        Res(\sigma_0, \{ \textsc{Peek} \}) &= \{ \sigma_0, \sigma_1, \sigma_2 \} \\
        \\
        Res_0(\sigma_0, \{ \textsc{Pet} \}) &= \{ \sigma_0 \} = Res(\sigma_0, \{ \textsc{Pet} \}) \\
        \\
        \Delta_1(\{\textsc{Peek},\textsc{Pet}\}, \sigma_0) &= \{\textsc{Peek}\} \\
        \Delta_2(\{\textsc{Peek},\textsc{Pet}\}, \sigma_0) &= \{\textsc{Pet}\} \\
        Res_0(\Delta_1(\{\textsc{Peek},\textsc{Pet}\}, \sigma_0), \sigma_0) &= Res_0(\sigma_0, \{ \textsc{Peek} \}) \\
        Res_0(\Delta_2(\{\textsc{Peek},\textsc{Pet}\}, \sigma_0), \sigma_0) &= Res_0(\sigma_0, \{ \textsc{Pet} \}) \\
        Res_0(\sigma_0, \{\textsc{Peek},\textsc{Pet}\}) &= Res_0(\Delta_1(\{\textsc{Peek},\textsc{Pet}\}, \sigma_0), \sigma_0) \cup Res_0(\Delta_2(\{\textsc{Peek},\textsc{Pet}\}, \sigma_0), \sigma_0) \\
        Res_0(\sigma_0, \{\textsc{Peek},\textsc{Pet}\}) &= \{ \sigma_0, \sigma_1, \sigma_2 \} \\
        New(\{\textsc{Peek},\textsc{Pet}\}, \sigma_0, \sigma_0) &=  \{ \textit{alive}, \textit{purring} \} \\
        New(\{\textsc{Peek},\textsc{Pet}\}, \sigma_0, \sigma_1) &=  \{ \textit{alive}, \neg \textit{purring} \} \\
        New(\{\textsc{Peek},\textsc{Pet}\}, \sigma_0, \sigma_2) &=  \{ \neg \textit{alive}, \neg \textit{purring} \} \\
        Res(\sigma_0, \{\textsc{Peek},\textsc{Pet}\}) &= \{ \sigma_0, \sigma_1, \sigma_2 \}
    \end{align*}
    
    W stanie $\sigma_0$ akcje $\textsc{Peek}, \textsc{Pet}$ są konfliktowe, ponieważ obie wpływają na fluent \textit{purring}. Dlatego dla akcji złożonej $\{\textsc{Peek}, \textsc{Pet}\}$ wyróżnione są dwie dekompozycje $\Delta_1$ i $\Delta_{2}$.
    
    \item Dla~stanu $\sigma_1$ mamy
    \begin{align*}
        Res_0(\sigma_1, \{ \textsc{Peek} \}) &= \{ \sigma_0, \sigma_1, \sigma_2 \} \\
        New(\{ \textsc{Peek} \}, \sigma_1, \sigma_0) &=  \{ \textit{alive}, \textit{purring} \} \\
        New(\{ \textsc{Peek} \}, \sigma_1, \sigma_1) &=  \{ \textit{alive}, \neg \textit{purring} \} \\
        New(\{ \textsc{Peek} \}, \sigma_1, \sigma_2) &=  \{ \neg \textit{alive}, \neg \textit{purring} \} \\
        Res(\sigma_1, \{ \textsc{Peek} \}) &= \{ \sigma_0, \sigma_1, \sigma_2 \} \\
        \\
        Res_0(\sigma_1, \{ \textsc{Pet} \}) &= \{ \sigma_0 \} = Res(\sigma_1, \{ \textsc{Pet} \}) \\
        \\
        \Delta_1(\{\textsc{Peek},\textsc{Pet}\}, \sigma_1) &= \{\textsc{Peek}\} \\
        \Delta_2(\{\textsc{Peek},\textsc{Pet}\}, \sigma_1) &= \{\textsc{Pet}\} \\
        Res_0(\Delta_1(\{\textsc{Peek},\textsc{Pet}\}, \sigma_1), \sigma_1) &= Res_0(\sigma_1, \{ \textsc{Peek} \}) \\
        Res_0(\Delta_2(\{\textsc{Peek},\textsc{Pet}\}, \sigma_1), \sigma_1) &= Res_0(\sigma_1, \{ \textsc{Pet} \}) \\
        Res_0(\sigma_1, \{\textsc{Peek},\textsc{Pet}\}) &= Res_0(\Delta_1(\{\textsc{Peek},\textsc{Pet}\}, \sigma_1), \sigma_1) \cup Res_0(\Delta_2(\{\textsc{Peek},\textsc{Pet}\}, \sigma_1), \sigma_1) \\
        Res_0(\sigma_1, \{\textsc{Peek},\textsc{Pet}\}) &= \{ \sigma_0, \sigma_1, \sigma_2 \} \\
        New(\{\textsc{Peek},\textsc{Pet}\}, \sigma_1, \sigma_0) &=  \{ \textit{alive}, \textit{purring} \} \\
        New(\{\textsc{Peek},\textsc{Pet}\}, \sigma_1, \sigma_1) &=  \{ \textit{alive}, \neg \textit{purring} \} \\
        New(\{\textsc{Peek},\textsc{Pet}\}, \sigma_1, \sigma_2) &=  \{ \neg \textit{alive}, \neg \textit{purring} \} \\
        Res(\sigma_1, \{\textsc{Peek},\textsc{Pet}\}) &= \{ \sigma_0, \sigma_1, \sigma_2 \}
    \end{align*}
    
    \item Dla~stanu $\sigma_2$ mamy
    \begin{align*}
        Res_0(\sigma_2, \{ \textsc{Peek} \}) &= \{ \sigma_2 \} = Res(\sigma_2, \{ \textsc{Peek} \}) \\
        \\
        Res_0(\sigma_2, \{ \textsc{Pet} \}) &= \{ \sigma_2 \} = Res(\sigma_2, \{ \textsc{Pet} \}) \\
        \\
        \Delta_1(\{\textsc{Peek},\textsc{Pet}\}, \sigma_2) &= \{\textsc{Peek}\} \\
        \Delta_2(\{\textsc{Peek},\textsc{Pet}\}, \sigma_2) &= \{\textsc{Pet}\} \\
        Res_0(\Delta_1(\{\textsc{Peek},\textsc{Pet}\}, \sigma_2), \sigma_2) &= Res_0(\sigma_2, \{ \textsc{Peek} \}) \\
        Res_0(\Delta_2(\{\textsc{Peek},\textsc{Pet}\}, \sigma_2), \sigma_2) &= Res_0(\sigma_2, \{ \textsc{Pet} \}) \\
        Res_0(\sigma_2, \{\textsc{Peek},\textsc{Pet}\}) &= Res_0(\Delta_1(\{\textsc{Peek},\textsc{Pet}\}, \sigma_2), \sigma_2) \cup Res_0(\Delta_2(\{\textsc{Peek},\textsc{Pet}\}, \sigma_2), \sigma_2) \\
        Res_0(\sigma_2, \{\textsc{Peek},\textsc{Pet}\}) &= \{ \sigma_2 \} = Res(\sigma_2, \{\textsc{Peek},\textsc{Pet}\})
    \end{align*}
\end{itemize}

\begin{figure}[H]
    \centering
    \begin{tikzpicture}[->,node distance=8cm]
        \tikzstyle{state} = [shape=rectangle,draw]
        \tikzstyle{initial} = [fill=gray!25]
        \node [initial,state] (s0) {$\sigma_0 = \{ \textit{alive}, \textit{purring} \}$};
        \node [right of=s0,below=4cm of s0,state] (s1) {$\sigma_2 = \{ \neg \textit{alive}, \neg \textit{purring} \}$};
        \node [below=of s0,initial,state] (s2) {$\sigma_1 = \{ \textit{alive}, \neg \textit{purring} \}$};

        \path (s0)  edge node [above right, align=left] {$\{ \textsc{Peek} \}$ \\ $\{ \textsc{Peek, Pet} \}$} (s1)
                    edge [out=90, in=180, looseness=2] node [above left, align=right] {$\{ \textsc{Peek} \}$ \\ $\{ \textsc{Pet} \}$ \\ $\{ \textsc{Peek}, \textsc{Pet} \}$} (s0)
                    edge [bend left=15] node [right, align=left] {$\{ \textsc{Peek} \}$ \\ $\{ \textsc{Peek, Pet} \}$} (s2);
        \path (s1)  edge [out=90, in=360, looseness=2] node [above right, align=left] {$\{ \textsc{Peek} \}$ \\ $ \{ \textsc{Pet} \}$ \\ $ \{ \textsc{Peek}, \textsc{Pet} \}$} (s1);
        \path (s2)  edge [bend left=15] node [left, align=right] {$\{ \textsc{Peek} \}$ \\ $\{ \textsc{Pet} \}$ 
        \\ $\{ \textsc{Peek}, \textsc{Pet} \}$} (s0)
                    edge [out=270, in=180, looseness=2] node [below left, align=right] {$\{ \textsc{Peek} \}$ \\ $\{ \textsc{Peek}, \textsc{Pet} \}$} (s2)
                    edge node [below right, align=left] {$\{ \textsc{Peek} \}$ \\ $\{ \textsc{Peek, Pet} \}$} (s1);
    \end{tikzpicture}
    \caption{
        Graf funkcji przejścia dla~scenariusza \emph{Kot Schrödingera}.
        Kolorem szarym oznaczone potencjalne stany początkowe.
    }
    \label{fig:producer-consumer-state}
\end{figure}

\newpage
Przykładowe kwerendy nad~$\mathcal{D}$:

\begin{enumerate}
    \item Niech dana będzie kwerenda
    $$ \Possibly \neg \textit{alive} \After \{ \textsc{Pet} \}, \{ \textsc{Peek} \} $$
    Kwerenda ta \textbf{będzie} konsekwencją $\mathcal{D}$, ponieważ dla~modelu~$\mathcal{D}$, którego stanem początkowym jest~$\sigma_0$ i dla modelu którego stanem początkowym jest $\sigma_1$, istnieje $\Psi_\mathcal{S}$, że program $(\{ \textsc{Pet} \}, \{ \textsc{Peek} \})$ prowadzi do stanu $\sigma_2$, w którym spełnione jest $\neg \textit{alive}$.

    \item Niech dana będzie kwerenda
    $$ \Necessary \textit{purring} \After \{ \textsc{Peek}, \textsc{Pet} \} $$
    Kwerenda ta \textbf{nie~będzie} konsekwencją $\mathcal{D}$, ponieważ w~modelu~$\mathcal{D}$, którego stanem początkowym jest~$\sigma_0$ lub $\sigma_1$, dla programu $(\{ \textsc{Peek}, \textsc{Pet} \})$ istnieje ścieżka $\Psi_\mathcal{S}$ prowadząca   do stanu $\sigma_2$.

    \item Kwerendy $\Executable \Sometimes$ oraz $\Executable \Always$ \textbf{są}~konsekwencjami $\mathcal{D}$ dla~każdego programu $\mathcal{P}$, ponieważ dowolny program da~się wykonać dla~każdego modelu $\mathcal{D}$ z dowolnym stanem początkowym, ze względu na brak zdań $\Impossible$.

    \item Niech dana będzie kwerenda
    $$ \Accessible \textit{purring} $$
    Kwerenda ta \textbf{jest} konsekwencją $\mathcal{D}$, ponieważ dla modelu $\mathcal{D}$, którego stanem początkowym jest $\sigma_0$, pusty program prowadzi do osiągnięcia celu. Z kolei dla modelu, którego stanem poczatkowym jest $\sigma_1$, program $(\{\textsc{Pet}\})$ prowadzi do osiągnięcia ceul.
\end{enumerate}

\subsection{Dwóch malarzy}
Załóżmy, że mamy dwóch malarzy (oznaczmy ich jako $A$ i $B$) i jeden pędzel. Malarze mogą zabrać pędzel (akcje $\textsc{TakeA},\textsc{TakeB}$), na początku żaden z malarzy nie ma pędzla. Malarze mający pędzel mogą wykonać akcję $\textsc{Paint}$, po której znów odkładają pędzel.

Powyższy scenariusz jest reprezentowany przez następującą domenę akcji $\mathcal{D}$:

\begin{align*}
    & \Initially  \neg \textit{brushA}, \neg \textit{brushB} \\
	& \textsc{TakeA} \Causes \textit{brushA} \land \neg \textit{brushB}  \\
	& \textsc{TakeB} \Causes \neg \textit{brushA} \land \textit{brushB}  \\
	& \textsc{Paint} \Causes \neg \textit{brushA} \If \textit{brushA} \\
	& \textsc{Paint} \Causes \neg \textit{brushB} \If \textit{brushB} \\
	& \Always \neg \textit{brushA} \lor \neg \textit{brushB}
\end{align*}

Rozważmy po~kolei wykonanie wszystkich możliwych akcji złożonych w~każdym stanie. Możliwe~są trzy stany:

\begin{enumerate}
    \item $\sigma_0 = \{ \neg \textit{brushA}, \neg \textit{brushB} \}$,
    \item $\sigma_1 = \{ \textit{brushA}, \neg \textit{brushB} \}$,
    \item $\sigma_2 = \{ \neg \textit{brushA}, \textit{brushB} \}$.
\end{enumerate}

Stanem początkowym będzie $\sigma_0$.

\begin{itemize}
    \item Dla~stanu $\sigma_0$ mamy
    
    \begin{align*}
		Res_0(\sigma_0, \{ \textsc{TakeA} \}) &= \{ \sigma_1 \} = Res(\sigma_0, \{ \textsc{TakeA} \}) \\
        \\
    	Res_0(\sigma_0, \{ \textsc{TakeB} \}) &= \{ \sigma_2 \} = Res(\sigma_0, \{ \textsc{TakeB} \}) \\
    	\\
    	Res_0(\sigma_0, \{ \textsc{Paint} \}) &= \{ \sigma_0 \} = Res(\sigma_0, \{ \textsc{Paint} \}) \\
    	\\
    	Res_0(\sigma_0, \Delta_1(\{\textsc{TakeA, TakeB} \}, \sigma_0 \})) &= Res_0(\sigma_0, \{ \textsc{TakeA} \}) = \{ \sigma_1 \} \\
    	Res_0(\sigma_0, \Delta_2(\{\textsc{TakeA, TakeB} \}, \sigma_0 \})) &= Res_0(\sigma_0, \{ \textsc{TakeB} \}) = \{ \sigma_2 \} \\
		Res_0(\sigma_0, \{\textsc{TakeA, TakeB} \}) &= \{ \sigma_1, \sigma_2 \} \\  
        New(\{\textsc{TakeA, TakeB} \}, \sigma_0, \sigma_1) &=  \{ \textit{brushA} \} \\    
		New(\{\textsc{TakeA, TakeB} \}, \sigma_0, \sigma_2) &=  \{ \textit{brushB} \} \\    
        Res(\sigma_0, \{\textsc{TakeA, TakeB} \}) &= \{ \sigma_1, \sigma_2 \} \\   
        \\ 
        Res_0(\sigma_0, \{ \textsc{TakeA, Paint} \}) &= \{ \sigma_1\} = Res(\sigma_0, \{ \textsc{TakeA, Paint} \}) \\ 
        \\   
        Res_0(\sigma_0, \{ \textsc{TakeB, Paint} \}) &= \{ \sigma_2\} = Res(\sigma_0, \{ \textsc{TakeB, Paint} \}) \\     
        \\
        Res_0(\sigma_0, \Delta_1(\{\textsc{TakeA, TakeB, Paint} \}, \sigma_0 \})) &= Res_0(\sigma_0, \{ \textsc{TakeA, Paint} \}) = \{ \sigma_1 \} \\
    	Res_0(\sigma_0, \Delta_2(\{\textsc{TakeA, TakeB, Paint} \}, \sigma_0 \})) &= Res_0(\sigma_0, \{ \textsc{TakeB, Paint} \}) = \{ \sigma_2 \} \\
		Res_0(\sigma_0, \{\textsc{TakeA, TakeB, Paint} \}) &= \{ \sigma_1, \sigma_2 \} \\         
        New(\{\textsc{TakeA, TakeB, Paint} \}, \sigma_0, \sigma_1) &=  \{ \textit{brushA} \} \\    
		New(\{\textsc{TakeA, TakeB, Paint} \}, \sigma_0, \sigma_2) &=  \{ \textit{brushB} \} \\    
        Res(\sigma_0, \{\textsc{TakeA, TakeB, Paint} \}) &= \{ \sigma_1, \sigma_2 \} \\ 		
    \end{align*}

	W tym stanie akcje $\textsc{TakeA i TakeB}$ są konfliktowe, fluenty \textit{brushA, brushB} są sprzeczne, dlatego dla akcji zawierających $\textsc{TakeA i TakeB}$ trzeba wyznaczyć dekompozycje.
	\newpage
    \item Dla~stanu $\sigma_1$ mamy
    
    \begin{align*}
		Res_0(\sigma_1, \{ \textsc{TakeA} \}) &= \{ \sigma_1 \} = Res(\sigma_1, \{ \textsc{TakeA} \}) \\
        \\
    	Res_0(\sigma_1, \{ \textsc{TakeB} \}) &= \{ \sigma_2 \} = Res(\sigma_1, \{ \textsc{TakeB} \}) \\
    	\\
    	Res_0(\sigma_1, \{ \textsc{Paint} \}) &= \{ \sigma_0, \sigma_2 \}  \\
    	New(\{ \textsc{Paint} \}, \sigma_1, \sigma_0) &=  \{ \textit{brushA} \} \\    
    	New(\{ \textsc{Paint} \}, \sigma_1, \sigma_2) &=  \{ \textit{brushA}, \textit{brushB}\} \\   
    	Res(\sigma_1, \{ \textsc{Paint} \}) &= \{ \sigma_0 \} \\ 
    	\\
    	Res_0(\sigma_1, \Delta_1(\{\textsc{TakeA, TakeB} \}, \sigma_1 \})) &= Res_0(\sigma_1, \{ \textsc{TakeA} \}) = \{ \sigma_1 \} \\
    	Res_0(\sigma_1, \Delta_2(\{\textsc{TakeA, TakeB} \}, \sigma_1 \})) &= Res_0(\sigma_1, \{ \textsc{TakeB} \}) = \{ \sigma_2 \} \\
		Res_0(\sigma_1, \{\textsc{TakeA, TakeB} \}) &= \{ \sigma_1, \sigma_2 \} \\  
        New(\{\textsc{TakeA, TakeB} \}, \sigma_1, \sigma_1) &=  \emptyset \\    
		New(\{\textsc{TakeA, TakeB} \}, \sigma_1, \sigma_2) &=  \{ \textit{brushA}, \textit{brushB} \} \\    
        Res(\sigma_0, \{\textsc{TakeA, TakeB} \}) &= \{ \sigma_1 \} \\   
        \\             
    	Res_0(\sigma_1, \Delta_1(\{\textsc{TakeA, Paint} \}, \sigma_1 \})) &= Res_0(\sigma_1, \{ \textsc{TakeA} \}) = \{ \sigma_1 \} \\
    	Res_0(\sigma_1, \Delta_2(\{\textsc{TakeA, Paint} \}, \sigma_1 \})) &= Res_0(\sigma_1, \{ \textsc{Paint} \}) = \{ \sigma_0, \sigma_2 \} \\
		Res_0(\sigma_1, \{\textsc{TakeA, Paint} \}) &= \{ \sigma_0, \sigma_1, \sigma_2 \} \\  
        New(\{ \textsc{TakeA, Paint} \}, \sigma_1, \sigma_0) &= \{ \textit{brushA}\} \\    
		New(\{ \textsc{TakeA, Paint} \}, \sigma_1, \sigma_1) &= \emptyset \\  
 		New(\{ \textsc{TakeA, Paint} \}, \sigma_1, \sigma_2) &= \{ \textit{brushA}, \textit{brushB} \} \\  
        Res(\sigma_1, \{\textsc{TakeA, Paint} \}) &= \{ \sigma_1 \} \\   
        \\     
    	Res_0(\sigma_1, \{ \textsc{TakeB, Paint} \})) &=  \{ \sigma_2 \} = Res(\sigma_1, \{ \textsc{TakeB, Paint} \})) \\
        \\    
    	Res_0(\sigma_1, \Delta_1(\{\textsc{TakeA, TakeB, Paint} \}, \sigma_1 \})) &= Res_0(\sigma_1, \{ \textsc{TakeA} \}) = \{ \sigma_1 \} \\
    	Res_0(\sigma_1, \Delta_2(\{\textsc{TakeA, TakeB, Paint} \}, \sigma_1 \})) &= Res_0(\sigma_1, \{ \textsc{TakeB, Paint} \}) = \{ \sigma_2 \} \\
		Res_0(\sigma_1, \{\textsc{TakeA, TakeB, Paint} \}) &= \{ \sigma_1, \sigma_2 \} \\  
        New(\{ \textsc{TakeA, TakeB, Paint} \}, \sigma_1, \sigma_1) &= \emptyset \\    
		New(\{ \textsc{TakeA, TakeB, Paint} \}, \sigma_1, \sigma_2) &= \{ \textit{brushA}, \textit{brushB}\} \\    
        Res(\sigma_1, \{\textsc{TakeA, TakeB, Paint} \}) &= \{ \sigma_1 \} \\   
        \\               
    \end{align*}
    
    W stanie $\sigma_1$ dodatkowo $\textsc{TakeA, Paint}$ są sprzeczne, ustawiają fluent \textit{BrushA} na przeciwne wartości.
	\newpage
    \item Dla~stanu $\sigma_2$ mamy
    
    \begin{align*}
		Res_0(\sigma_2, \{ \textsc{TakeA} \}) &= \{ \sigma_1 \} = Res(\sigma_2, \{ \textsc{TakeA} \}) \\
        \\    	
    	Res_0(\sigma_2, \{ \textsc{TakeB} \}) &= \{ \sigma_2 \} = Res(\sigma_2, \{ \textsc{TakeB} \}) \\
    	\\
    	Res_0(\sigma_2, \{ \textsc{Paint} \}) &= \{ \sigma_0, \sigma_1 \}  \\
    	New(\{ \textsc{Paint} \}, \sigma_2, \sigma_0) &=  \{ \textit{brushB} \} \\    
    	New(\{ \textsc{Paint} \}, \sigma_2, \sigma_1) &=  \{ \textit{brushA}, \textit{brushB}\} \\   
    	Res(\sigma_2, \{ \textsc{Paint} \}) &= \{ \sigma_0 \} \\ 
    	\\     
    	Res_0(\sigma_2, \Delta_1(\{\textsc{TakeA, TakeB} \}, \sigma_2 \})) &= Res_0(\sigma_1, \{ \textsc{TakeA} \}) = \{ \sigma_1 \} \\
    	Res_0(\sigma_2, \Delta_2(\{\textsc{TakeA, TakeB} \}, \sigma_2 \})) &= Res_0(\sigma_1, \{ \textsc{TakeB} \}) = \{ \sigma_2 \} \\
		Res_0(\sigma_2, \{\textsc{TakeA, TakeB} \}) &= \{ \sigma_1, \sigma_2 \} \\  
        New(\{\textsc{TakeA, TakeB} \}, \sigma_2, \sigma_1) &=  \{ \textit{brushA}, \textit{brushB} \} \\    
		New(\{\textsc{TakeA, TakeB} \}, \sigma_2, \sigma_2) &=  \emptyset \\    
        Res(\sigma_2, \{\textsc{TakeA, TakeB} \}) &= \{ \sigma_2 \} \\
        \\
    	Res_0(\sigma_2, \{ \textsc{TakeA, Paint} \})) &=  \{ \sigma_1 \} = Res(\sigma_2, \{ \textsc{TakeA, Paint} \})) \\
        \\ 
    	Res_0(\sigma_2, \Delta_1(\{\textsc{TakeB, Paint} \}, \sigma_2 \})) &= Res_0(\sigma_2, \{ \textsc{TakeB} \}) = \{ \sigma_2 \} \\
    	Res_0(\sigma_2, \Delta_2(\{\textsc{TakeB, Paint} \}, \sigma_2 \})) &= Res_0(\sigma_2, \{ \textsc{Paint} \}) = \{ \sigma_0, \sigma_1 \} \\
		Res_0(\sigma_2, \{\textsc{TakeB, Paint} \}) &= \{ \sigma_0, \sigma_1, \sigma_2 \} \\  
        New(\{ \textsc{TakeB, Paint} \}, \sigma_2, \sigma_0) &= \{ \textit{brushB}\} \\    
		New(\{ \textsc{TakeB, Paint} \}, \sigma_2, \sigma_1) &= \{ \textit{brushA}, \textit{brushB} \} \\  
 		New(\{ \textsc{TakeB, Paint} \}, \sigma_2, \sigma_2) &= \emptyset \\  
        Res(\sigma_2, \{\textsc{TakeB, Paint} \}) &= \{ \sigma_2 \} \\   
        \\ 
    	Res_0(\sigma_2, \Delta_1(\{\textsc{TakeA, TakeB, Paint} \}, \sigma_2 \})) &= Res_0(\sigma_2, \{ \textsc{TakeA, Paint} \}) = \{ \sigma_1 \} \\
    	Res_0(\sigma_2, \Delta_2(\{\textsc{TakeA, TakeB, Paint} \}, \sigma_2 \})) &= Res_0(\sigma_2, \{ \textsc{TakeB} \}) = \{ \sigma_2 \} \\
		Res_0(\sigma_2, \{\textsc{TakeA, TakeB, Paint} \}) &= \{ \sigma_0, \sigma_1, \sigma_2 \} \\  
        New(\{ \textsc{TakeA, TakeB, Paint} \}, \sigma_2, \sigma_0) &= \{ \textit{brushB}\} \\    
		New(\{ \textsc{TakeA, TakeB, Paint} \}, \sigma_2, \sigma_1) &= \{ \textit{brushA}, \textit{brushB} \} \\  
 		New(\{ \textsc{TakeA, TakeB, Paint} \}, \sigma_2, \sigma_2) &= \emptyset \\  
        Res(\sigma_2, \{\textsc{TakeA, TakeB, Paint} \}) &= \{ \sigma_2 \} \\   
        \\                   
    \end{align*}
\end{itemize}

\begin{figure}[H]
    \hspace*{-4em}
    \begin{tikzpicture}[->,node distance=9cm]
        \tikzstyle{state} = [shape=rectangle,draw]
        \tikzstyle{initial} = [fill=gray!25]
        \node [initial,state] (s0) {$\sigma_0 = \{ \neg \textit{brushA}, \neg \textit{brushB} \}$};
        \node [right of=s0 ,state] (s1) {$\sigma_1 = \{ \textit{brushA}, \neg \textit{brushB} \}$};
        \node [below=of s0,state] (s2) {$\sigma_2 = \{ \neg \textit{brushA}, \textit{brushB} \}$};

        \path (s0)  edge [out=90, in=180, looseness=2] node [above left, align=left] {$\{ \textsc{Paint} \}$ } (s0) 
        edge [out=360, in=180, bend left=20] node [above, align=left] {$\{ \textsc{TakeA} \}$, \\ $\{ \textsc{TakeA, TakeB} \}$, \\ $\{ \textsc{TakeA, Paint} \}$, \\ $\{ \textsc{TakeA, TakeB, Paint} \}$ } (s1)
        edge [out=270, in=90, bend right=15] node [above left, align=right] {$\{ \textsc{TakeB} \}$, \\ $\{ \textsc{TakeA, TakeB} \}$, \\ $\{ \textsc{TakeB, Paint} \}$, \\ $\{ \textsc{TakeA, TakeB, Paint} \}$ } (s2);  
        
        \path (s1)  edge [out=180, in=360] node [below right, align=left] {$\{ \textsc{Paint} \}$ } (s0)
        edge [out=360, in=90, looseness=2] node [above, align=left] {$\{ \textsc{TakeA} \}$, \\ $\{ \textsc{TakeA, TakeB} \}$, \\ $\{ \textsc{TakeA, Paint} \}$, \\ $\{ \textsc{TakeA, TakeB, Paint} \}$ } (s1)   
         edge node [above left, align=left] {$\{ \textsc{TakeB} \}$, \\ $\{ \textsc{TakeB, Paint} \}$ } (s2);    

        \path (s2)  edge [out=90, in=270] node [right, pos=0.75, align=left] {$\{ \textsc{Paint} \}$ } (s0)
         edge [out=360, in=270] node [below right, align=left] {$\{ \textsc{TakeA} \}$, \\ $\{ \textsc{TakeA, Paint} \}$ } (s1)
         edge [out=180, in=270, looseness=2] node [below, align=left] {$\{ \textsc{TakeB} \}$, \\ $\{ \textsc{TakeA, TakeB} \}$, \\ $\{ \textsc{TakeB, Paint} \}$,\\ $\{ \textsc{TakeA, TakeB, Paint} \}$  } (s2);   
    \end{tikzpicture}
    \caption{
        Graf funkcji przejścia dla~scenariusza \emph{Dwóch malarzy}.
        Kolorem szarym oznaczone potencjalne stany początkowe.
    }
    \label{fig:malarze}
\end{figure}

Przykładowe kwerendy nad~$\mathcal{D}$:

\begin{enumerate}
    \item Niech dana będzie kwerenda
    $$ \Possibly \textit{brushA} \After \{ \textsc{TakeA} , \textsc{TakeB} \} $$
    Kwerenda ta \textbf{będzie} konsekwencją $\mathcal{D}$, ponieważ dla~modelu~$\mathcal{D}$, którego stanem początkowym jest~$\sigma_0$, istnieje $\Psi_\mathcal{S}$, że program $(\{ \textsc{TakeA} , \textsc{TakeB} \})$ prowadzi do stanu $\sigma_1$, w którym spełnione jest $\textit{brushA}$.

    \item Niech dana będzie kwerenda
    $$ \Necessary \neg \textit{brushA} \land \neg \textit{brushB} \After \{ \textsc{TakeA}, \textsc{TakeB} \}, \{ \textsc{TakeA}, \textsc{TakeB}, \textsc{Paint} \} $$
    Kwerenda ta \textbf{nie~będzie} konsekwencją $\mathcal{D}$, ponieważ w~modelu~$\mathcal{D}$, którego stanem początkowym jest~$\sigma_0$, dla programu $(\{ \textsc{TakeA}, \textsc{TakeB} \}, \{ \textsc{TakeA}, \textsc{TakeB}, \textsc{Paint} \})$ nie istnieje ścieżka $\Psi_\mathcal{S}$ prowadząca do stanu $\sigma_0$.

    \item Kwerendy $\Executable \Sometimes$ oraz $\Executable \Always$ \textbf{są}~konsekwencjami $\mathcal{D}$ dla~każdego programu $\mathcal{P}$, ponieważ dowolny program da~się wykonać dla~modelu $\mathcal{D}$ ze stanem początkowym $\sigma_0$, ze względu na brak zdań $\Impossible$.

    \item Niech dana będzie kwerenda
    $$ \Accessible  \textit{brushA} \land \textit{brushB} $$
    Kwerenda ta \textbf{nie będzie} konsekwencją $\mathcal{D}$, ponieważ dla modelu $\mathcal{D}$, którego stanem początkowym jest $\sigma_0$, nie istnieje program, który prowadzi do osiągnięcia celu. Cel jest sprzeczny ze~zdefiniowaną domeną akcji $\mathcal{D}$.
\end{enumerate}

\end{document}